\documentclass{beamer}
\mode<presentation>{
  \usetheme{Madrid}
  \usecolortheme{crane}
  %\setbeamertemplate{footline} % To remove the footer line in all slides uncomment this line
  %\setbeamertemplate{footline}[page number] % To replace the footer line in all slides with a simple slide count uncomment this line
  \setbeamertemplate{navigation symbols}{} % To remove the navigation symbols from the bottom of all slides uncomment this line
}
\setbeamercolor{itemize item}{fg=red}
\usepackage{listings}
\usepackage{caption}
\usepackage{setspace}
\usepackage{ragged2e} % For justifying text
\usepackage{multicol}
\usepackage{multirow}
\usepackage{graphicx} % Allows including images
\usepackage{booktabs} % Allows the use of \toprule, \midrule and \bottomrule in tables
\usepackage{enumerate}
\usepackage{amsmath}
\usepackage[final]{pdfpages}
\usepackage{hyperref}
\usepackage{pgffor}
\usepackage[makeroom]{cancel}
\usepackage{xcolor,colortbl}
\usefonttheme[onlymath]{serif}
\definecolor{RoyalYellow}{RGB}{250,218,94}
\graphicspath{{/afs/cern.ch/user/h/htong/www/}} % Specifies the directory where pictures are stored
\newcommand{\tabincell}[2]{\begin{tabular}{@{}#1@{}}#2\end{tabular}} 
\hypersetup{colorlinks=true}
\setbeamertemplate{title page}{
  \vbox{}
  \begingroup
  \centering{
    \usebeamercolor[fg]{titlegraphic}\inserttitlegraphic\par}\vskip0.5em
  \begin{beamercolorbox}[sep=8pt,center]{title}
    \usebeamerfont{title}\inserttitle\par%
    \ifx\insertsubtitle\@empty%
    \else%
    \vskip0.25em{%
      \usebeamerfont{subtitle}\usebeamercolor[fg]{subtitle}\insertsubtitle\par}%
    \fi%
  \end{beamercolorbox}%
  \vskip1em\par
  \begin{beamercolorbox}[sep=8pt,center]{author}
    \usebeamerfont{author}\insertauthor
  \end{beamercolorbox}
  \begin{beamercolorbox}[sep=8pt,center]{institute}
    \usebeamerfont{institute}\insertinstitute
  \end{beamercolorbox}
  \begin{beamercolorbox}[sep=8pt,center]{date}
    \usebeamerfont{date}\insertdate
  \end{beamercolorbox}
  \endgroup
  \vfill
}

% TITLE PAGE

% logo of my university
\titlegraphic{\includegraphics[width=2cm]{cms.jpg}\hspace*{8cm}~%
  \includegraphics[width=2cm]{ncu.jpg}
}

\title[]{Update for toy-MC test in $Z' \rightarrow ZH \rightarrow llbb$} % The short title appears at the bottom of every slide, the full title is only on the title page
\author[Henry Tong]{Yee-Shian Henry Tong} % Your name
\institute[NCU]{ % Your institution as it will appear on the bottom of every slide, may be shorthand to save space
  National Central University \\ % Your institution for the title page
  \medskip
  \textit{NCU HEP Group Meeting} % Your email address
}
\date{\today} % Date, can be changed to a custom date

\begin{document}

\begin{frame}
  \vspace*{-1cm}
  \enlargethispage{1cm}
  \titlepage % Print the title page as the first slide
\end{frame}

% PRESENTATION SLIDES

\begin{frame}
  \frametitle{toy-MC test}
  \justifying
  \begin{footnotesize}
    \begin{itemize}
    \item Split e and mu channels (combine e and mu before)
    \item Cut at ZH mass $>$ 750 GeV (cut at 500 GeV before)
    \item Use sidebands 30-65 GeV, 135-300 GeV (40-65 GeV, 145-inf GeV before)
    \item Apply 1 and 2 b-tags on the subjet (no apply any b-tags before)
    \item For the b-tags selection, also need to be split in two categories.
    \item Not use isPassLoose now, use HEEPNoIso+miniIso in the Z-related selection.
    \item Need to study separately for every single background (DY+jets, ttbar, dibosons, single-top)
    \item In summary, there are four sets of results for four kinds of backgrounds.
    \end{itemize}
  \end{footnotesize}
\end{frame}

\begin{frame}
  \Huge{\centerline{Electron channel, DY+jets}}
\end{frame}


\foreach \n/\m in {1/1,2/2,3/3,4/4}{
  \begin{frame}
    \frametitle{Electron channel, DY+jets}
    Left: deltaR $<$ 0.3, subjet $>$ 0; Right: deltaR $>$ 0.3, subjet $>$ 1
    \begin{tabular}{ll}
      \includegraphics[page=\n,width=0.45\textwidth]{rooFittoyMCResults/rooFit_toyMC_ele_Zjets_cat1.pdf} &
      \includegraphics[page=\m,width=0.45\textwidth]{rooFittoyMCResults/rooFit_toyMC_ele_Zjets_cat2.pdf} \\
    \end{tabular}
  \end{frame}
}

\begin{frame}
  \Huge{\centerline{Electron channel, Dibosons}}
\end{frame}

\foreach \n/\m in {1/1,2/2,3/3,4/4}{
  \begin{frame}
    \frametitle{Electron channel, Dibosons}
    Left: deltaR $<$ 0.3, subjet $>$ 0; Right: deltaR $>$ 0.3, subjet $>$ 1
    \begin{tabular}{ll}
      \includegraphics[page=\n,width=0.45\textwidth]{rooFittoyMCResults/rooFit_toyMC_ele_VV_cat1.pdf} &
      \includegraphics[page=\m,width=0.45\textwidth]{rooFittoyMCResults/rooFit_toyMC_ele_VV_cat2.pdf} \\
    \end{tabular}
  \end{frame}
}



\begin{frame}
  \Huge{\centerline{Electron channel, TT}}
\end{frame}

\foreach \n/\m in {1/1,2/2,3/3,4/4}{
  \begin{frame}
    \frametitle{Electron channel, TT}
    Left: deltaR $<$ 0.3, subjet $>$ 0; Right: deltaR $>$ 0.3, subjet $>$ 1
    \begin{tabular}{ll}
      \includegraphics[page=\n,width=0.45\textwidth]{rooFittoyMCResults/rooFit_toyMC_ele_TT_cat1.pdf} &
      \includegraphics[page=\m,width=0.45\textwidth]{rooFittoyMCResults/rooFit_toyMC_ele_TT_cat2.pdf} \\
    \end{tabular}
  \end{frame}
}

\begin{frame}
  \Huge{\centerline{Muon channel, DY+jets}}
\end{frame}

\foreach \n/\m in {1/1,2/2,3/3,4/4}{
  \begin{frame}
    \frametitle{Muon channel, DY+jets}
    Left: deltaR $<$ 0.3, subjet $>$ 0; Right: deltaR $>$ 0.3, subjet $>$ 1
    \begin{tabular}{ll}
      \includegraphics[page=\n,width=0.45\textwidth]{rooFittoyMCResults/rooFit_toyMC_mu_Zjets_cat1.pdf} &
      \includegraphics[page=\m,width=0.45\textwidth]{rooFittoyMCResults/rooFit_toyMC_mu_Zjets_cat2.pdf} \\
    \end{tabular}
  \end{frame}
}

\begin{frame}
  \Huge{\centerline{Muon channel, Dibosons}}
\end{frame}

\foreach \n/\m in {1/1,2/2,3/3,4/4}{
  \begin{frame}
    \frametitle{Muon channel, Dibosons}
    Left: deltaR $<$ 0.3, subjet $>$ 0; Right: deltaR $>$ 0.3, subjet $>$ 1
    \begin{tabular}{ll}
      \includegraphics[page=\n,width=0.45\textwidth]{rooFittoyMCResults/rooFit_toyMC_mu_VV_cat1.pdf} &
      \includegraphics[page=\m,width=0.45\textwidth]{rooFittoyMCResults/rooFit_toyMC_mu_VV_cat2.pdf} \\
    \end{tabular}
  \end{frame}
}



\begin{frame}
  \Huge{\centerline{Muon channel, TT}}
\end{frame}

\foreach \n/\m in {1/1,2/2,3/3,4/4}{
  \begin{frame}
    \frametitle{Muon channel, TT}
    Left: deltaR $<$ 0.3, subjet $>$ 0; Right: deltaR $>$ 0.3, subjet $>$ 1
    \begin{tabular}{ll}
      \includegraphics[page=\n,width=0.45\textwidth]{rooFittoyMCResults/rooFit_toyMC_mu_TT_cat1.pdf} &
      \includegraphics[page=\m,width=0.45\textwidth]{rooFittoyMCResults/rooFit_toyMC_mu_TT_cat2.pdf} \\
    \end{tabular}
  \end{frame}
}

\begin{frame}
   \frametitle{Summary}
   \justifying
   \begin{footnotesize}
     \begin{itemize}
     \item Start to study estimate the background in data set. All the fitting procedure are being studied using RooFit.
     \item For minor backgrounds such as VV and TT (ignore single top), subtract them directly from data set (according to MC). Will assign systematic uncertainty to this assumption.
     \end{itemize}
   \end{footnotesize}
 \end{frame}

\end{document}
