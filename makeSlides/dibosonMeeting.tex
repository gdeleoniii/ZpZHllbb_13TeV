
\documentclass{beamer}
\mode<presentation>{
  \usetheme{Madrid}
  \usecolortheme{crane}
  %\setbeamertemplate{footline} % To remove the footer line in all slides uncomment this line
  %\setbeamertemplate{footline}[page number] % To replace the footer line in all slides with a simple slide count uncomment this line
  \setbeamertemplate{navigation symbols}{} % To remove the navigation symbols from the bottom of all slides uncomment this line
}
\setbeamercolor{itemize item}{fg=red}
\usepackage{listings}
\usepackage{caption}
\usepackage{setspace}
\usepackage{ragged2e} % For justifying text
\usepackage{multicol}
\usepackage{multirow}
\usepackage{graphicx} % Allows including images
\usepackage{booktabs} % Allows the use of \toprule, \midrule and \bottomrule in tables
\usepackage{enumerate}
\usepackage{amsmath}
\usepackage[final]{pdfpages}
\usepackage{hyperref}
\usepackage{pgffor}
\usepackage[makeroom]{cancel}
\usepackage{xcolor,colortbl}
\usefonttheme[onlymath]{serif}
\definecolor{RoyalYellow}{RGB}{250,218,94}
\graphicspath{{/afs/cern.ch/user/h/htong/www/}} % Specifies the directory where pictures are stored
\newcommand{\tabincell}[2]{\begin{tabular}{@{}#1@{}}#2\end{tabular}} 
\hypersetup{colorlinks=true}
\setbeamertemplate{title page}{
  \vbox{}
  \begingroup
  \centering{
    \usebeamercolor[fg]{titlegraphic}\inserttitlegraphic\par}\vskip0.5em
  \begin{beamercolorbox}[sep=8pt,center]{title}
    \usebeamerfont{title}\inserttitle\par%
    \ifx\insertsubtitle\@empty%
    \else%
    \vskip0.25em{%
      \usebeamerfont{subtitle}\usebeamercolor[fg]{subtitle}\insertsubtitle\par}%
    \fi%
  \end{beamercolorbox}%
  \vskip1em\par
  \begin{beamercolorbox}[sep=8pt,center]{author}
    \usebeamerfont{author}\insertauthor
  \end{beamercolorbox}
  \begin{beamercolorbox}[sep=8pt,center]{institute}
    \usebeamerfont{institute}\insertinstitute
  \end{beamercolorbox}
  \begin{beamercolorbox}[sep=8pt,center]{date}
    \usebeamerfont{date}\insertdate
  \end{beamercolorbox}
  \endgroup
  \vfill
}

% TITLE PAGE

% logo of my university
\titlegraphic{\includegraphics[width=2cm]{cms.jpg}\hspace*{8cm}~%
  \includegraphics[width=2cm]{ncu.jpg}
}

\title[]{Toy-MC test in $Z' \rightarrow ZH \rightarrow llbb$} % The short title appears at the bottom of every slide, the full title is only on the title page
\author[Henry Tong]{Yee-Shian Henry Tong} % Your name
\institute[NCU]{ % Your institution as it will appear on the bottom of every slide, may be shorthand to save space
  National Central University \\ % Your institution for the title page
  \medskip
  \textit{Diboson Meeting} % Your email address
}
\date{\today} % Date, can be changed to a custom date

\begin{document}

\begin{frame}
  \vspace*{-1cm}
  \enlargethispage{1cm}
  \titlepage % Print the title page as the first slide
\end{frame}

% PRESENTATION SLIDES

\begin{frame}
  \frametitle{1000 toy-MC test: Method}
  \justifying
  \begin{footnotesize}
    \begin{itemize}
    \item Fit the pruned mass distribution (40-240 GeV) in MC (*)
    \item Generate 1000 toy-MC samples by filling histogram with random data from original fitting function (from the pruned mass distribution (40-240 GeV) in pesudo-data). The total event number is equal to the number as pseudo-data. This is in order to make the shape consistent.
    \item Dig out the signal region from the 1000 samples, left only side band region for fitting in next step.
    \item Fit the 1000 toy-MC samples by fixing the "shape" parameters get from (*), left only one degree of freedom (which is normalization parameter).
    \item Calculate the bias in signal region (105 - 135 GeV) from the formula:
      \begin{itemize}
      \item bias = ( estimate events from fitting curve - true event from histogram ) / true event from histogram.
      \end{itemize}
    \end{itemize}
  \end{footnotesize}
\end{frame}

\begin{frame}
  \frametitle{1000 toy-MC test: Fitting function}
  \justifying
  \begin{footnotesize}
    \begin{itemize}
    \item We have two fitting functions to deal with the pruned mass which is with or without signal region.
    \item For the pruned mass with signal region,
      \[f(x) = \frac{p_0\exp(p_1 x)\frac{1+Erf(\frac{x-p_2}{p_3})}{2}}{p_4 K_0}\]
      \begin{itemize}
      \item $K_0$ is the integrated area from 40-240 GeV.
      \end{itemize}
    \item For the pruned mass without signal region,
      \[f(x) = \frac{p_0\exp(p_1 x)\frac{1+Erf(\frac{x-p_2}{p_3})}{2}}{p_4(K_1+K_2)}\]
      \begin{itemize}
      \item $K_1$ is the integrated area from 40-65 GeV. 
      \item $K_2$ is the integrated area from 145-240 GeV.
      \end{itemize}
    \item $p_4$ is the histogram bin width.
    \end{itemize}
  \end{footnotesize}
\end{frame}

\begin{frame}
  \frametitle{Preselections}
  \begin{center}
    \begin{tiny}
      \begin{tabular}[t]{ | l | l | }
        \hline
        HLT       
        & HLT\_Ele105\_CaloIdVT\_GsfTrkIdT\_* or HLT\_Mu45\_eta2p1\_*  \\
        \hline
        nVtx 
        & $>=$ 1                                         \\
        \hline
        \multirow{7}{*}{Electrons} 
        & $|\eta|$ $<$ 2.5                               \\
        & remove 1.4442 $<$ $|\eta|$ $<$ 1.566           \\
        & HEEP NoIso ID                                  \\
        & rho-corrected mini Iso $<$ 0.1                 \\
        & Leading electron $p_T$ $>$ 115 GeV             \\
        & Sub-leading electron $p_T$ $>$ 35 GeV          \\
        & Two electrons have opposite charge             \\
        \hline
        \multirow{8}{*}{Muons}
        & Custom Teacker Muon ID                         \\ 
        & One of both pass High $p_T$ Muon ID            \\
        & mini Iso $<$ 0.2                               \\
        & Muon $p_T$ $>$ 20 GeV                          \\
        & Muon $|\eta|$ $<$ 2.4                          \\
        & At least one muon $p_T$ $>$ 50 GeV             \\
        & At least one muon $|\eta|$ $<$ 2.1             \\
        & Two muons have opposite charge                 \\
        \hline
        \multirow{2}{*}{Z bosons}
        & 70GeV $<$ $M_{ll}$ $<$ 110 GeV                 \\
        & $p_{Tll}$ $>$ 200 GeV                          \\
        \hline
        \multirow{6}{*}{AK8 jets}
        & $|\eta|$ $<$ 2.4                               \\ 
        & $p_T$ $>$ 200 GeV                              \\
        & FATjetPassIDLoose                              \\
        & $\Delta R$ between jet and lepton $>$ 0.8      \\
        & side band: 40 GeV $<$ corrPRmass $<$ 65 GeV, corrPRmass $>$ 145 GeV \\
        & signal region: 105 GeV $<$ corrPRmass $<$ 135 GeV \\
        \hline
        $ZH$ mass                 
        & $>$ 500 GeV                                    \\
        \hline      
      \end{tabular}
      \captionof{table}{\footnotesize{Preselections for the background estimation.}}
      \label{tab:select}
    \end{tiny}
  \end{center}
\end{frame}

\begin{frame}
  \frametitle{1000 toy-MC test: Pruned mass distribution}
  \begin{center}
    \begin{footnotesize}
      \begin{itemize}
      \item We use the parameters from the fit curve of right plot to fit the 1000 samples of pruned mass (without signal region) in pseudo-data.
      \end{itemize}
    \end{footnotesize}
    \begin{tabular}{ll}
      \includegraphics[width=0.45\textwidth,page=1]{aRanoorg.pdf} &
      \includegraphics[width=0.45\textwidth,page=2]{aRanoorg.pdf} \\
    \end{tabular}
    \captionof{table}{\footnotesize{Pruned mass distribution of (side band + signal region) in pseudo-data (left) and in MC (right).}}
  \end{center}
\end{frame}

\begin{frame}
  \frametitle{1000 toy-MC test: Pruned mass distribution}
  \begin{center}
    \begin{tabular}{ll}
      \includegraphics[width=0.45\textwidth,page=1]{aRanoorg.pdf} &
      \includegraphics[width=0.45\textwidth,page=3]{aRanoorg.pdf} \\
    \end{tabular}
    \captionof{table}{\footnotesize{Pruned mass distribution of (side band + signal region)(left) and of side band region only (right) in pseudo-data.}}
  \end{center}
\end{frame}

\begin{frame}
  \frametitle{1000 toy-MC test: Comparison of bias}
  \begin{center}
    \begin{footnotesize}
      \begin{itemize}
      \item Below are the bias distribution for different event numbers in pruned mass. If we have more events in pseudo-data, we can get a better bias. Red line is fit curve using Gaussian function.
      \end{itemize}
    \end{footnotesize}
    \begin{tabular}{ll}
      \includegraphics[width=0.45\textwidth,page=5]{aRanoorg.pdf} &
      \includegraphics[width=0.45\textwidth,page=4]{aRano1e5.pdf} \\
    \end{tabular}
    \captionof{table}{\footnotesize{Left: Bias with the same event number as data. (Mean=-0.019 $\sigma$=0.079); Right: Bias with event number = 100000. (Mean=-0.00013, $\sigma$=0.010)}}
  \end{center}
\end{frame}

\begin{frame}
  \frametitle{1000 toy-MC test: Comparison of bias (original fitting method)}
  \begin{center}
    \begin{footnotesize}
      \begin{itemize}
      \item The original fit means fixing the normalization parameter, let 3 shape parameters float freely when fit the pruned mass.
      \item Below are the bias distribution for different event numbers in pruned mass. The results are worse compare with the fitting method in previous slides. Red line is fit curve using Gaussian function.
      \end{itemize}
    \end{footnotesize}
    \begin{tabular}{ll}
      \includegraphics[width=0.45\textwidth,page=4]{aRorgorg.pdf} &
      \includegraphics[width=0.45\textwidth,page=4]{aRorg1e5.pdf} \\
    \end{tabular}
    \captionof{table}{\footnotesize{Left: Bias with the same event number as data. (Mean=-0.053 $\sigma$=0.18); Right: Bias with event number = 100000. (Mean=0.016, $\sigma$=0.035)}}
  \end{center}
\end{frame}

\begin{frame}
  \frametitle{Conclusions}
  \begin{center}
    \begin{footnotesize}
      \begin{itemize}
      \item The bias is reduced when the number of events in each toy-MC sample is large.
      \item The bias is smaller when fixing the shape parameters and the size is about 2\% when the number of events in each toy MC is equivalent to the number of events in data.
      \end{itemize}
    \end{footnotesize}
  \end{center}
\end{frame}

\end{document}
