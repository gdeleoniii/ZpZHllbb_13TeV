%%%%%%%%%%%%%%%%%%%%%%%%%%%%%%%%%%%%%%%
%
% a0poster Portrait Poster
% For 2016 PSROC Annual Meeting
%
%%%%%%%%%%%%%%%%%%%%%%%%%%%%%%%%%%%%%%%

%----------------------------------------------------------------------------------------
%	PACKAGES AND OTHER DOCUMENT CONFIGURATIONS
%----------------------------------------------------------------------------------------

\documentclass[a0,portrait]{a0poster}
\usepackage[fontsize=30pt]{scrextend}
\usepackage{multicol}         % This is so we can have multiple columns of text side-by-side
\columnsep=90pt              % This is the amount of white space between the columns in the poster
\usepackage[svgnames]{xcolor} % Specify colors by their 'svgnames', for a full list of all colors available see here: http://www.latextemplates.com/svgnames-colors
\usepackage{times}
\usepackage{graphicx}         % Required for including images
\graphicspath{{./}}     % Location of the graphics files
\usepackage{booktabs}         % Top and bottom rules for table
\usepackage[font=normalsize,labelfont=bf]{caption} % Required for specifying captions to tables and figures
\usepackage{amsfonts, amsmath, amsthm, amssymb}    % For math fonts, symbols and environments
\usepackage{wrapfig}          % Allows wrapping text around tables and figures
\usepackage{setspace}
\usepackage{multicol}
\usepackage{multirow}
\usepackage{color,colortbl}
%\usepackage{indentfirst}

\renewcommand{\baselinestretch}{1.2} % Line spacing 
\newcommand{\tabincell}[2]{\begin{tabular}{@{}#1@{}}#2\end{tabular}} 

\setlength{\parindent}{1.5cm}
\setlength{\parskip}{0.5cm}

\begin{document}

%----------------------------------------------------------------------------------------
%	POSTER HEADER 
%----------------------------------------------------------------------------------------

\makebox[72cm][c]{ %%
  \begin{minipage}[1]{0.1\linewidth}
    \includegraphics[scale=1.3]{ncu.png}\\
  \end{minipage}
  %
  \begin{minipage}[l]{0.8\linewidth}
    \begin{center}
      \veryHuge \color{FireBrick} \textbf{Background estimation in the search for $Z' \rightarrow ZH \rightarrow llbb$ at $\sqrt{s}$ = 13 TeV with the CMS detector} \color{Black}\\[1cm] % Title
      \Large \textbf{Yee-Shian Henry Tong, Shin-Shan Eiko Yu, Raman Khurana, Yun-Ju Lu, Yu-Hsiang Chang, Jun-Yi Wu, Ji-Kong Huang, Ching-Wei Chen}\\ % Author(s)
      \Large Department of Physics, National Central University\\ % University/organization
    \end{center}
  \end{minipage}
  %
  \begin{minipage}[1]{0.1\linewidth}
    \includegraphics[scale=1.3]{cms.jpg}\\
  \end{minipage}
} %%

\vspace{2cm} % A bit of extra whitespace between the header and poster content

%----------------------------------------------------------------------------------------

\begin{multicols}{2} % This is how many columns your poster will be broken into, a portrait poster is generally split into 2 columns


  %----------------------------------------------------------------------------------------

  \color{Black} % Navy color for the abstract

  \section*{\color{Crimson} Abstract}

  We present the study of how to compare the predicted SM background with the observed data for $Z' \rightarrow ZH \rightarrow llbb$ at $\sqrt{s}$ = 13 TeV. We define the signal region, and side-band region as a pure background control region by using pruned jet mass distribution. In order to estimate the final background, we consider the $m_{ZH}$ MC mass spectrum in the signal and side-band region. A ratio $\alpha (m_{ZH})$ of the two allows a prediction of the mass spectrum in the signal region starting from the measured distribution in the side-band. Under the assumption that this extrapolation from the side-band to the signal region works in the same way both for data and MC, we can estimate the final background distribution by multiplying the $m_{ZH}$ mass spectrum observed in the side-band by this $\alpha$ ratio.

  %----------------------------------------------------------------------------------------

  \section*{\color{Crimson} Pre-selections}
  
  Table reports a summary of the pre-selection requirement in this study.
  
  \begin{center}
    \begin{small}
      \begin{tabular}[t]{ | l | l | }
        \hline
        HLT       
        & HLT\_Ele105\_CaloIdVT\_GsfTrkIdT\_* or HLT\_Mu45\_eta2p1\_*  \\
        \hline
        nVtx 
        & $>=$ 1                                         \\
        \hline
        \multirow{7}{*}{Electrons} 
        & $|\eta|$ $<$ 2.5                               \\
        & remove 1.4442 $<$ $|\eta|$ $<$ 1.566           \\
        & HEEP NoIso ID                                  \\
        & mini Iso $<$ 0.1                               \\
        & Leading electron $p_T$ $>$ 115 GeV             \\
        & Sub-leading electron $p_T$ $>$ 35 GeV          \\
        & Two electrons have opposite charge             \\
        \hline
        \multirow{8}{*}{Muons}
        & Custom Teacker Muon ID                         \\ 
        & One of both pass High $p_T$ Muon ID            \\
        & mini Iso $<$ 0.2                               \\
        & Muon $p_T$ $>$ 20 GeV                          \\
        & Muon $|\eta|$ $<$ 2.4                          \\
        & At least one muon $p_T$ $>$ 50 GeV             \\
        & At least one muon $|\eta|$ $<$ 2.1             \\
        & Two muons have opposite charge                 \\
        \hline
        \multirow{2}{*}{Z bosons}
        & 70GeV $<$ $M_{ll}$ $<$ 110 GeV                 \\
        & $p_{Tll}$ $>$ 200 GeV                          \\
        \hline
        \multirow{6}{*}{AK8 jets}
        & $|\eta|$ $<$ 2.4                               \\ 
        & $p_T$ $>$ 200 GeV                              \\
        & FATjetPassIDLoose                              \\
        & $\Delta R$ between jet and lepton $>$ 0.8      \\
        & side band: 40 GeV $<$ corrPRmass $<$ 65 GeV, corrPRmass $>$ 145 GeV \\
        & signal region: 105 GeV $<$ corrPRmass $<$ 135 GeV \\
        \hline
        $ZH$ mass                 
        & $>$ 500 GeV                                    \\
        \hline      
      \end{tabular}
      \captionof{table}{\footnotesize{Preselections for the background estimation.}}
    \end{small}
  \end{center}

  %----------------------------------------

  \section*{\color{Crimson} Pruned mass distribution}

  \begin{center}
    \includegraphics[page=2,width=0.3\textwidth]{alphaRatio.pdf}
    \captionof{figure}{\color{DarkRed} The pruned mass distribution without signal region (black) and its fitting curve (red). ~\cite{PDF}.}
  \end{center}
  
  %----------------------------------------

  \section*{\color{Crimson} $Z'$ mass distribution in MC}

  \begin{center}
    \begin{tabular}{ll}
      \includegraphics[page=4,width=0.225\textwidth]{alphaRatio.pdf} &
      \includegraphics[page=5,width=0.225\textwidth]{alphaRatio.pdf} \\
    \end{tabular}
    \captionof{figure}{\color{DarkRed} The mass distribution of $Z'$ in signal region (left) and side band (right) of MC. ~\cite{PDF}.}
  \end{center}

  %----------------------------------------
  
  \section*{\color{Crimson} The $\alpha$ ratio}
  
  The $\alpha$ ratio is a ratio of number of events in $M_{ZH}$ distribution.
  \begin{equation}
    \alpha(M_{ZH}) = \frac{N^{MC,bkg}_{signal}(M_{ZH})}{N^{MC,bkg}_{side}(M_{ZH})}
  \end{equation}
  Choose corrected pruned mass of $H (\rightarrow b\bar{b})$ jets as control variable, in order to decide signal region and side band.
  To account the background in signal region of pseudo-data:
  \begin{equation}
    N_{bkg}(M_{ZH}) = N^{pseudo\ data}_{side-band}(M_{ZH})\times \alpha(M_{ZH})
  \end{equation}
  It is better to use data (pseudo-data) itself to constrain the normalization.
  The approch is to fit the side-bands of the distribution of the pruned jet mass observed in data (pseudo-data) by using
  \begin{equation}
    f_{ErfExp}(x) = p_0\exp(p_1x)\frac{1+Erf(\frac{x-p_2}{p_3})}{2}
  \end{equation}

  \begin{center}
    \includegraphics[page=6,width=0.3\textwidth]{alphaRatio.pdf}
    \captionof{figure}{\color{DarkRed} The $\alpha$ ratio of MC. ~\cite{PDF}.}
  \end{center}

  %----------------------------------------
  
  \section*{\color{Crimson} Number of backgrounds in signal region}

  Number of backgrounds in signal region is 172 + 18 - 23.

  \begin{center}
    \includegraphics[page=7,width=0.3\textwidth]{alphaRatio.pdf}
    \captionof{figure}{\color{DarkRed} The mass distribution of $ZH$ in signal region of pseudo-data (red). ~\cite{PDF}.}
  \end{center}
  
  %----------------------------------------------------------------------------------------

  \section*{\color{Crimson} Forthcoming Research}

  %\color{DimGray}

  %----------------------------------------------------------------------------------------
  %	REFERENCES
  %----------------------------------------------------------------------------------------

  %  \nocite{*} % Print all references regardless of whether they were cited in the poster or not
  % \bibliographystyle{unsrt} % Unsort referencing style
  %\renewcommand\refname{\color{Crimson} References}
  %\bibliography{references} % Use the example bibliography file references.bib

  %----------------------------------------------------------------------------------------

\end{multicols}
\end{document}

% End of poster
