\documentclass[7pt,aspectratio=1610]{beamer}
\mode<presentation>{
  \usetheme{Madrid}
  \usecolortheme{crane}
  %\setbeamertemplate{footline} % To remove the footer line in all slides uncomment this line
  %\setbeamertemplate{footline}[page number] % To replace the footer line in all slides with a simple slide count uncomment this line
  \setbeamertemplate{navigation symbols}{} % To remove the navigation symbols from the bottom of all slides uncomment this line
}

\setbeamersize{text margin left=6pt,text margin right=6pt}
\usepackage{listings}
\usepackage{caption}
\usepackage{setspace}
\usepackage{ragged2e} % For justifying text
\usepackage{multicol}
\usepackage{multirow}
\usepackage{graphicx} % Allows including images
\usepackage{booktabs} % Allows the use of \toprule, \midrule and \bottomrule in tables
\usepackage{enumerate}
\usepackage{lmodern}
\usepackage[final]{pdfpages}
\usepackage{amsmath}
\usepackage{hyperref}
\usepackage{pgffor}
\usepackage[makeroom]{cancel}
\usepackage{xcolor,colortbl}
\graphicspath{{/afs/cern.ch/user/h/htong/www/}} % Specifies the directory where pictures are stored
\newcommand{\tabincell}[2]{\begin{tabular}{@{}#1@{}}#2\end{tabular}}
\renewcommand{\arraystretch}{1.2}
\hypersetup{colorlinks=true}
\setbeamertemplate{title page}{
  \vbox{}
  \begingroup
  \centering{
    \usebeamercolor[fg]{titlegraphic}\inserttitlegraphic\par}\vskip0.15em
  \begin{beamercolorbox}[sep=7pt,center]{title}
    \usebeamerfont{title}\inserttitle\par%
    \ifx\insertsubtitle\@empty%
    \else%
    \vskip0.25em{%
      \usebeamerfont{subtitle}\usebeamercolor[fg]{subtitle}\insertsubtitle\par}%
    \fi%
  \end{beamercolorbox}%
  \vskip1em\par
  \begin{beamercolorbox}[sep=8pt,center]{author}
    \usebeamerfont{author}\insertauthor
  \end{beamercolorbox}
  \begin{beamercolorbox}[sep=8pt,center]{institute}
    \usebeamerfont{institute}\insertinstitute
  \end{beamercolorbox}
  \begin{beamercolorbox}[sep=8pt,center]{date}
    \usebeamerfont{date}\insertdate
  \end{beamercolorbox}
  \endgroup
  \vfill
}

% TITLE PAGE

\titlegraphic{\includegraphics[width=1cm]{cms.jpg}\hspace*{9cm}~%
  \includegraphics[width=1cm]{ncu.jpg}
}

\title[]{\LARGE{Background studies and preliminary expected limit for the $Z' \rightarrow ZH \rightarrow llbb$ analysis}} % The short title appears at the bottom of every slide, the full title is only on the title page
\author[Henry Tong]{\Large{Yee-Shian Henry Tong}\\[3mm] \scriptsize{Shin-Shan Eiko Yu, Raman Khurana, Yun-Ju Lu, Yu-Hsiang Chang, Jun-Yi Wu, Ji-Kong Huang, Ching-Wei Chen}} % Your name
\institute[NCU]{ % Your institution as it will appear on the bottom of every slide, may be shorthand to save space
  National Central University \\ % Your institution for the title page
  \medskip
  \textit{Dibosons Resonance Meeting}
}
\date{\today} % Date, can be changed to a custom date

\begin{document}

\begin{frame}
  \vspace*{-1cm}
  \enlargethispage{1cm}
  \titlepage % Print the title page as the first slide
\end{frame}


% PRESENTATION SLIDES

\everymath{\displaystyle}
\renewcommand{\arraystretch}{1.25}

\begin{frame}
  \frametitle{Monte-Carlo Samples}
  \justifying 
  \begin{tiny}
    \begin{center}
      \begin{tabular}{ | l | l | }
        \hline
        \bf SPRING15 25-ns samples & \bf Cross sections (pb) \\
        \hline
        DYJetsToLL\_M-50\_HT-100to200\_13TeV-madgraphMLM-pythia8
        & 147.4 $\times$ 1.23  \\
        \hline
        DYJetsToLL\_M-50\_HT-200to400\_13TeV-madgraphMLM-pythia8
        & 40.99 $\times$ 1.23  \\
        \hline
        DYJetsToLL\_M-50\_HT-400to600\_13TeV-madgraphMLM-pythia8
        & 5.678 $\times$ 1.23  \\
        \hline
        DYJetsToLL\_M-50\_HT-600toInf\_13TeV-madgraphMLM-pythia8
        & 2.198 $\times$ 1.23   \\
        \hline  
        TT\_TuneCUETP8M1\_13TeV-powheg-pythia8MC25ns
        & 831.76      \\
        \hline
        WW\_TuneCUETP8M1\_13TeV-pythia8MC25ns 
        & 118.7       \\
        \hline
        WZ\_TuneCUETP8M1\_13TeV-pythia8MC25ns 
        & 47.13       \\
        \hline
        ZZ\_TuneCUETP8M1\_13TeV-pythia8MC25ns 
        & 16.523      \\
        \hline
      \end{tabular}
      \captionof{table}{\tiny{MC samples used and corresponding cross section values (K-factors for DY samples are included for normalization). The MC samples are miniAOD V2.}}
    \end{center}
  \end{tiny}
\end{frame}

\begin{frame}
  \LARGE{\centerline{Study of backgrounds by using MC}}
\end{frame}
\begin{frame}
  \frametitle{Introduction}
  \justifying
  \begin{footnotesize}
    This study is to test the fitting and $\alpha$ method by splitting the MC samples into one half, one is used to build $\alpha$ ratio and the other is used as
    pseudo sideband data and used to perform fits.
  \end{footnotesize}
\end{frame}
\begin{frame}
  \frametitle{Procedure of doing $\alpha$ method}
  \justifying
  \begin{footnotesize}
    The $\alpha$ ratio is a ratio of number of events in $M_{ZH}$ distribution.
    \begin{equation}
      \alpha(M_{ZH}) = \frac{N^{MC,bkg}_{signal}(M_{ZH})}{N^{MC,bkg}_{side}(M_{ZH})}
    \end{equation}
    Choose corrected pruned mass of $H (\rightarrow b\bar{b})$ jets as control variable, in order to decide signal region and side band.
    To account the background in signal region of pseudo-data:
    \begin{equation}
      N_{bkg}(M_{ZH}) = N^{pseudo\ data}_{sideband}(M_{ZH})\times \alpha(M_{ZH})
    \end{equation}
    It is better to use data (pseudo-data) itself to constrain the normalization.
    The approch is to fit the sidebands of the distribution of the pruned jet mass observed in data (pseudo-data) by using
    \begin{equation}
      f_{ErfExp}(x) = p_0\exp(p_1x)\frac{1+Erf(\frac{x-p_2}{p_3})}{2}
    \end{equation}
  \end{footnotesize}
\end{frame}
\renewcommand{\arraystretch}{1.1}
\begin{frame}
  \frametitle{Preselections}
  \justifying
  \begin{tiny}
    \begin{center}
      \begin{tabular}[t]{ | l | l | }
        \hline
        HLT       
        & HLT\_Ele105\_CaloIdVT\_GsfTrkIdT\_* or HLT\_Mu45\_eta2p1\_*  \\
        \hline
        nVtx 
        & $>=$ 1                                         \\
        \hline
        \multirow{7}{*}{Electrons} 
        & $|\eta|$ $<$ 2.5                               \\
        & remove 1.4442 $<$ $|\eta|$ $<$ 1.566           \\
        & HEEP NoIso ID                                  \\
        & rho-corrected mini Iso $<$ 0.1                 \\
        & Leading electron $p_T$ $>$ 115 GeV             \\
        & Sub-leading electron $p_T$ $>$ 35 GeV          \\
        & Two electrons have opposite charge             \\
        \hline
        \multirow{8}{*}{Muons}
        & Custom Teacker Muon ID                         \\ 
        & One of both pass High $p_T$ Muon ID            \\
        & rho-corrected mini Iso $<$ 0.2                 \\
        & Muon $p_T$ $>$ 20 GeV                          \\
        & Muon $|\eta|$ $<$ 2.4                          \\
        & At least one muon $p_T$ $>$ 50 GeV             \\
        & At least one muon $|\eta|$ $<$ 2.1             \\
        & Two muons have opposite charge                 \\
        \hline
        \multirow{2}{*}{Z bosons}
        & 70GeV $<$ $M_{ll}$ $<$ 110 GeV                 \\
        & $p_{Tll}$ $>$ 200 GeV                          \\
        \hline
        \multirow{6}{*}{AK8 jets}
        & $|\eta|$ $<$ 2.4                               \\ 
        & $p_T$ $>$ 200 GeV                              \\
        & FATjetPassIDLoose                              \\
        & $\Delta R$ between jet and lepton $>$ 0.8      \\
        & side band: 40 GeV $<$ corrPRmass $<$ 65 GeV, corrPRmass $>$ 145 GeV \\
        & signal region: 105 GeV $<$ corrPRmass $<$ 135 GeV \\
        \hline
        $ZH$ mass                 
        & $>$ 500 GeV                                    \\
        \hline      
      \end{tabular}
      \captionof{table}{\footnotesize{Preselections for the toy MC study.}}
    \end{center}   
  \end{tiny}
\end{frame}
\begin{frame}
  \frametitle{Corrected pruned mass distribution}
  \begin{center}
    \begin{tabular}{ll}
      \includegraphics[page=1,width=0.45\textwidth]{alphaRatio.pdf}  &
      \includegraphics[page=1,width=0.45\textwidth]{roofitCheck.pdf} \\
    \end{tabular}
    \captionof{table}{\footnotesize{Left: $\chi^{2}$ fitting, Right: Using RooFit package}}
  \end{center}
\end{frame}
\begin{frame}
  \frametitle{Corrected pruned mass without signal region}
  \begin{center}
    \begin{tabular}{ll}
      \includegraphics[page=2,width=0.45\textwidth]{alphaRatio.pdf} &
      \includegraphics[page=3,width=0.45\textwidth]{alphaRatio.pdf} \\
    \end{tabular}
    \captionof{table}{\footnotesize{Number of backgrounds in signal region: 174.51 + 26.7675 - 27.8504}}
  \end{center}
\end{frame}
\begin{frame}
  \frametitle{Fitting tests}
  \begin{center}
    \begin{tabular}{l}
    \includegraphics[page=4,width=0.5\textwidth]{alphaRatio.pdf} \\
    \end{tabular}
    \captionof{table}{\footnotesize{The difference between the true number of background events within the signal region of the pruned mass and the results from the fit. The fitting tests are perform by generating 800 sets of pseudo-data based on MC.}}
  \end{center}
\end{frame}
\begin{frame}
  \frametitle{$ZH$ mass in signal region and side band of MC}
  \begin{center}
    \begin{tabular}{ll}
      \includegraphics[page=5,width=0.45\textwidth]{alphaRatio.pdf} &
      \includegraphics[page=6,width=0.45\textwidth]{alphaRatio.pdf} \\
    \end{tabular}
  \end{center}
\end{frame}
\begin{frame}
  \frametitle{$\alpha$ ratio of MC}
  \begin{center}
    \includegraphics[page=7,width=0.8\textwidth]{alphaRatio.pdf}
  \end{center}
\end{frame}
\begin{frame}
  \frametitle{$ZH$ mass in signal region of pseudo-data}
  \begin{center}
    \includegraphics[page=8,width=0.8\textwidth]{alphaRatio.pdf}
  \end{center}
\end{frame}
\begin{frame}
  \Large{\centerline{Study of limit setting in muon channel using shape analysis}}
\end{frame}
\renewcommand{\arraystretch}{1.2}
\begin{frame}
  \frametitle{Preselections}
  \justifying
  \begin{tiny}
    \begin{center}
      \begin{tabular}[t]{ | l | l | }
        \hline
        HLT
        & HLT\_Mu45\_eta2p1\_*                           \\
        \hline
        nVtx                   
        & $>=$ 1                                         \\
        \hline
        \multirow{8}{*}{Muons}
        & Custom Teacker Muon ID                         \\
        & One of both pass High $p_T$ Muon ID            \\
        & rho-corrected mini Iso $<$ 0.2                 \\
        & Muon $p_T$ $>$ 20 GeV                          \\
        & Muon $|\eta|$ $<$ 2.4                          \\
        & At least one muon $p_T$ $>$ 50 GeV             \\
        & At least one muon $|\eta|$ $<$ 2.1             \\

        & Two muons have opposite charge                 \\
        \hline
        \multirow{2}{*}{Z bosons}
        & 70GeV $<$ $M_{ll}$ $<$ 110 GeV                 \\
        & $p_{Tll}$ $>$ 200 GeV                          \\
        \hline
        \multirow{6}{*}{AK8 jets}
        & $|\eta|$ $<$ 2.4                               \\
        & $p_T$ $>$ 200 GeV                              \\
        & FATjetPassIDLoose                              \\
        & $\Delta R$ between jet and lepton $>$ 0.8      \\
        & 105 GeV $<$ corrPRmass $<$ 135 GeV             \\
        & at least one subjetCSV $<$ 0.605               \\
        \hline
        $ZH$ mass             
        & $>$ 700 GeV                                    \\
        \hline
      \end{tabular}
      \captionof{table}{\footnotesize{Preselections for limit setting study in muon channel.}}
    \end{center}
  \end{tiny}
\end{frame}
\begin{frame}
  \frametitle{Event numbers}
  \justifying
  \begin{tiny}
    \begin{center}
      \begin{tabular}{ | l | l | }
        \hline
        DATA   & 0          \\
        DYJETS & 35.934     \\
        TTBAR  & 0.545936   \\
        WW     & 0          \\
        WZ     & 0.415515   \\
        ZZ     & 0.269519   \\
        M800   & 7.23001    \\
        M1000  & 5.97106    \\
        M1200  & 3.38069    \\
        M1400  & 1.87863    \\
        M1600  & 1.04601    \\
        M1800  & 0.583509   \\
        M2000  & 0.34752    \\
        M2500  & 0.0947009  \\
        M3000  & 0.0273144  \\
        M3500  & 0.00845009 \\
        M4000  & 0          \\
        \hline
      \end{tabular}
      \captionof{table}{\footnotesize{Event numbers of backgrounds and signal samples (normalize to L = 3 $fb^{-1}$).}}
    \end{center}
  \end{tiny}
\end{frame}
\begin{frame}
  \frametitle{Results}
  \begin{center}
    \includegraphics[width=0.7\textwidth]{zhllbbCountingAsymptotic.pdf}
  \end{center}
\end{frame}
\begin{frame}
  \frametitle{Conclusions and To-dos}
  \justifying 
  \begin{footnotesize}
    \begin{enumerate}[1.]
      \setcounter{enumi}{0} %{-1}
    \item A preliminary test of background estimation has been performed using MC samples without applying the subjet CSV cuts.
    \item Preliminary expected limits (cut and count) are shown. 
    \item Will perform more studies of expected limits with different b-tagging selections and use the shapes of $M_{ZH}$ instead of cut and count.
    \item Will perform more fitting tests by generating 100-1000 sets of pseudo-data based on MC.
    \item Will update with the same lepton selections as Jose.
    \end{enumerate}
  \end{footnotesize}
\end{frame}
\begin{frame}
  \Huge{\centerline{Thanks for your Listening!}}
\end{frame}
\end{document}
