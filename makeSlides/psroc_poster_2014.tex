%%%%%%%%%%%%%%%%%%%%%%%%%%%%%%%%%%%%%%%
%
% a0poster Portrait Poster
% For 2014 PSROC Annual Meeting
% Edited by Henry Tong
% ---2014/01/20---
%
%%%%%%%%%%%%%%%%%%%%%%%%%%%%%%%%%%%%%%%

%----------------------------------------------------------------------------------------
%	PACKAGES AND OTHER DOCUMENT CONFIGURATIONS
%----------------------------------------------------------------------------------------

\documentclass[a0,portrait]{a0poster}
\usepackage[fontsize=26.6pt]{scrextend}
\usepackage{multicol} % This is so we can have multiple columns of text side-by-side
\columnsep=100pt % This is the amount of white space between the columns in the poster
\usepackage[svgnames]{xcolor} % Specify colors by their 'svgnames', for a full list of all colors available see here: http://www.latextemplates.com/svgnames-colors
\usepackage{times}
\usepackage{graphicx} % Required for including images
\graphicspath{{figures/}} % Location of the graphics files
\usepackage{booktabs} % Top and bottom rules for table
\usepackage[font=normalsize,labelfont=bf]{caption} % Required for specifying captions to tables and figures
\usepackage{amsfonts, amsmath, amsthm, amssymb} % For math fonts, symbols and environments
\usepackage{wrapfig} % Allows wrapping text around tables and figures
\usepackage{setspace}
\usepackage{color,colortbl}
%\usepackage{indentfirst}

\renewcommand{\baselinestretch}{1.2} % Line spacing 
\newcommand{\tabincell}[2]{\begin{tabular}{@{}#1@{}}#2\end{tabular}} 

\setlength{\parindent}{1.5cm}
\setlength{\parskip}{0.5cm}

\begin{document}

%----------------------------------------------------------------------------------------
%	POSTER HEADER 
%----------------------------------------------------------------------------------------

\makebox[72cm][c]{ %%
\begin{minipage}[1]{0.125\linewidth}
\includegraphics[scale=0.7]{NCU_logo.png}\\
\end{minipage}
%
\begin{minipage}[l]{0.75\linewidth}
\begin{center}
\veryHuge \color{FireBrick} \textbf{The Calculation of Cross-Sections and Uncertainties for Physics Processes at 8 TeV} \color{Black}\\[1cm] % Title
\LARGE \textbf{Yee-Shian Henry Tong \& Shin-Shan Eiko Yu}\\ % Author(s)
\LARGE Department of Physics, National Central University\\ % University/organization
\end{center}
\end{minipage}
%
\begin{minipage}[1]{0.125\linewidth}
\includegraphics[scale=0.7]{CMS_logo.jpg}\\
\end{minipage}
} %%

\vspace{2cm} % A bit of extra whitespace between the header and poster content

%----------------------------------------------------------------------------------------

\begin{multicols}{2} % This is how many columns your poster will be broken into, a portrait poster is generally split into 2 columns

%----------------------------------------------------------------------------------------
%	ABSTRACT
%----------------------------------------------------------------------------------------

\color{Black} % Navy color for the abstract

\section*{\color{Crimson} Abstract}

The current database in the CMS generator group is lack of uncertainties for the theoretical cross sections of a few important processes in the center-of-mass energy at 8 TeV. In order to help physicists in their analyses, we calculate the cross-section values and their corresponding uncertainties using the Monte-Carlo FeMtobarn Processes (MCFM) program.

%----------------------------------------------------------------------------------------
%	CROSS-SECTION
%----------------------------------------------------------------------------------------

\section*{\color{Crimson} Cross-Section}

Cross-section is the most important parameter in experimental high energy physics. If a function  $g(\mu)$ represents the probability function of the parton carrying the portion of longitudinal momentum of the proton according to Parton Distribution Function (PDF) and $f(\mu)$ represents the probability for a pair of partons to have certain interaction (ex: $q\bar{q}\rightarrow W^+W^-$), the total cross-section $\sigma(\mu)$ is $\sigma(\mu)=g(\mu)f(\mu)$. Here the $\mu$ is the reference energy scale in the perturbative calculations, or can be imagined as the reference point of the Taylor expansion in the perturbative calculations. If we want to find new physics, we need to use the predicted cross-section to estimate the background contribution in experimental data. As we eliminate the background contribution, signal of new physics would emerge.

%----------------------------------------------------------------------------------------
%	WHAT IS MCFM
%----------------------------------------------------------------------------------------

\section*{\color{Crimson} What is MCFM ?}

MCFM (Monte Carlo for FeMtobarn processes) is a program written by theoretical physicists. MCFM program can calculate the cross-section of femtobarn-level processes, which might be a rare physics process in the experiment of hardon-hardon collider. We use MCFM program to calculate the cross-section and uncertainty of important physics processes.

The usage of MCFM program is very simple. We need to produce the data cards the program needs, and than the program will do the calculation automatically. The data card is a formatted text file that can set the parameters for the calculation of MCFM program. The user can choose the physics process they need according to the user manual ~\cite{MCFM}, and set the number of events, center-of-mass energy, range of mass of interest, PDF, and so on.

%----------------------------------------------------------------------------------------
%	PARTON DISTRIBUTION FUNCTION
%----------------------------------------------------------------------------------------

\section*{\color{Crimson} Parton Distribution Function (PDF)}

The Parton Distribution Function (PDF) is an important parameter in the calculation of cross-section. The PDF represents a probability density function of parton carries the portion of longitudinal momentum of the corresponding proton ({\bf Figure 1}). The proton is not an elementary particle, and it consists of many partons, which is a hypothesis particle that can be used to describe the structure and properties of scattering of hadron. Nowadays, the parton is commonly known as quarks and gluons. The collision of proton with proton is just like a collision of two pack of partons. PDF is not a theoretical distribution function, it is obtained from fitting the experimental data from previous to the present.

%----------------------------------------
%	FIGURE OF PDF
%----------------------------------------

\begin{center}
\includegraphics[width=0.6\linewidth]{PDF.eps}
\captionof{figure}{\color{DarkRed} An example of the parton distribution function from the CT10 global analysis ~\cite{PDF}.}
\end{center}
%----------------------------------------

%----------------------------------------------------------------------------------------
%	CROSS-SECTION AND UNCERTAINTY
%----------------------------------------------------------------------------------------

\section*{\color{Crimson} Estimation of Uncertainties}
By comparing the cross-section values calculated using different PDFs, we can get the PDF uncertainty that will be used in the estimation of background contribution. The scale uncertainty comes from the difference of the cross-section value due to the change of $\mu$ value (let the $\mu$ be 0.5$\mu_0$ or 2$\mu_0$, where $\mu_0$ is the central value).

We do the calculation of the cross-section and corresponding uncertainties for the diboson and top quark productions. In the physics process we choose, the bosons or top quarks will further decay to other particles, we need to divide results with the multiplication of branching ratio in order to get the real cross-section value.

%----------------------------------------------------------------------------------------
%	RESULTS AND CONCLUSIONS
%----------------------------------------------------------------------------------------

\section*{\color{Crimson} Results and Conclusions}

The central value of cross-section of each physics process is consistent with the official value in the CMS database. The uncertainties of each process will be included in the official database and are ready to be used by other physicists ({\bf Table 2}).

%----------------------------------------
%	TABLE OF FEYNMAN DIAGRAM
%----------------------------------------

\begin{footnotesize}
\begin{center}\vspace{0.5cm}

\setlength{\arrayrulewidth}{1.1pt}
\renewcommand{\arraystretch}{1.3}

\begin{tabular}{|c|c|c|c|}

\hline
\rowcolor{NavajoWhite}

$WW$ {\bf process} &
$WZ$ {\bf process} &
$ZZ$ {\bf process} &
$t\bar{t}$ {\bf process} \\

\hline
\includegraphics[scale=0.45]{ww.jpg} &
\includegraphics[scale=0.45]{wz.jpg} &
\includegraphics[scale=0.45]{zz.jpg} &
\includegraphics[scale=0.45]{tt.jpg} \\

\hline
\end{tabular}

\captionof{table}{\color{DarkRed} The representative Feynman diagrams of each physics process~\cite{FEYN}.}
\end{center}
\end{footnotesize}
%----------------------------------------

%----------------------------------------
%	TABLE OF DATA
%----------------------------------------

\begin{footnotesize}
\begin{center}\vspace{0.5cm}

\setlength{\tabcolsep}{11pt}
\setlength{\arrayrulewidth}{1.1pt} % Set the width of table line
\renewcommand{\arraystretch}{1}

\begin{tabular}{|c|c|c|c|c|c|c||c|}

\hline
\rowcolor{NavajoWhite}

\bf No. &
\bf Process &
\tabincell{c}{\bf Min $\mathrm{\bf M}_{ll}$ \\ \bf cut (GeV)} &
\tabincell{c}{\bf Dynamic \\ \bf scale} &
\tabincell{c}{\bf Central \\ \bf value (fb)} &
\tabincell{c}{\bf PDF \\ \bf uncertainty} &
\tabincell{c}{\bf Scale \\ \bf uncertainty} &
\tabincell{c}{\bf Cross \\ \bf section (pb)} \\

\hline
1 & $W^{+}W^{-}$ & 0 & no & 648.2 & 21.2 & 27.3 & 56.1 \\ 
2 & $W^{+}Z^{0}$ & 12 & no & 75.2 & 3.7 & 4.8 & 21.2 \\ 
3 & $W^{+}Z^{0}$ & 40 & no & 53.6 & 2.1 & 2.9 & 15.1 \\ 
4 & $W^{-}Z^{0}$ & 12 & no & 44.2 & 1.8 & 2.5 & 12.4 \\ 
5 & $W^{-}Z^{0}$ & 40 & no & 30.6 & 1.8 & 1.6 & 8.6 \\ 
6 & $Z^{0}Z^{0}$ & 12 & no & 38.7 & 1.9 & 1.5 & 17.1 \\ 
7 & $Z^{0}Z^{0}$ & 40 & no & 18.5 & 0.7 & 1.0 & 8.2 \\ 
8 & $t\bar{t}$ & 0 & $\mu_1$ & 1896.6 & 181.4 & 461.3 & 214.6 \\ 
9 & $t\bar{t}$ & 0 & $\mu_2$ & 2223.8 & 218.8 & 348.6 & 251.7 \\ 

\hline
\end{tabular}

\captionof{table}{\color{DarkRed} The value of cross-section and its corresponding uncertainty for each physics process.}
\end{center}
\end{footnotesize}

%----------------------------------------

%----------------------------------------
%	FIGURE OF CROSS-SECTION
%----------------------------------------

\begin{center}\vspace{0.5cm}
\includegraphics[width=0.6\linewidth]{010K_10E_08TeV_out.eps}
\captionof{figure}{\color{DarkRed} The product of the cross-section and the branching ratio for each physics process.}
\end{center}


%----------------------------------------------------------------------------------------
%	FORTHCOMING RESEARCH
%----------------------------------------------------------------------------------------

\section*{\color{Crimson} Forthcoming Research}

\begin{itemize}
\item We will calculate the cross-section of same physics processes for higher center-of-mass energy of colliding particles (13 TeV and 14 TeV).
\item The program will perform two runs of Monte-Carlo: once for pre-conditioning and then the final run to collect the total cross-section and fill histograms. We will use higher number of sweeps for each run.
\item For every sweep of Monte-Carlo, the number of events generated in the pre-conditioning stage and the final run are adjustable. The error estimate on a total cross-section will often be reasonable for a fairly small number of events, whereas accurate histograms will require a longer run. Due to this reason, we will also increase the number of events in future work.
\end{itemize}

%\color{DimGray}

%----------------------------------------------------------------------------------------
%	REFERENCES
%----------------------------------------------------------------------------------------

\nocite{*} % Print all references regardless of whether they were cited in the poster or not
\bibliographystyle{unsrt} % Unsort referencing style
\renewcommand\refname{\color{Crimson} References}
\bibliography{references} % Use the example bibliography file references.bib

%----------------------------------------------------------------------------------------
%	ACKNOWLEDGEMENTS
%----------------------------------------------------------------------------------------

\section*{\color{Crimson} Acknowledgements}

Many thanks to the authors of MCFM for solving the technical problem of the WZ process.

%----------------------------------------------------------------------------------------

\end{multicols}
\end{document}

% End of poster
