\documentclass[a0,portrait]{a0poster}

\usepackage[top=2.2cm, bottom=2.2cm, left=2.5cm, right=2.5cm]{geometry}
\usepackage{lmodern}
\usepackage[fontsize=29.5pt]{scrextend}
\usepackage{amsmath,xspace,wrapfig}
\usepackage[format=hang]{caption}
\usepackage{multicol,multirow,fancybox}
\usepackage{color,graphicx,subfigure}
\usepackage{pdfposter}
\usepackage[svgnames]{xcolor}
\newcommand{\tabincell}[2]{\begin{tabular}{@{}#1@{}}#2\end{tabular}}
\renewcommand{\baselinestretch}{1.15} % Line spacing 
\renewcommand{\arraystretch}{1.15} % Table line spacing
\renewcommand{\rmdefault}{phv} % Arial
\renewcommand{\sfdefault}{phv} % Arial

\pagecolor{DarkCyan}

\begin{document}
\begin{center}

  
  %%%%%%%%%%%%%%%%%%%%%%%%%%%%%%%%%%%%%%%%%%%%%%%%%%%%%%%%%%%%%%%%%%%%%%%%%%%%%%%%
  %                           Start of Header                                    %
  %%%%%%%%%%%%%%%%%%%%%%%%%%%%%%%%%%%%%%%%%%%%%%%%%%%%%%%%%%%%%%%%%%%%%%%%%%%%%%%%

  \begin{titlebox}
    
    \begin{minipage}[c]{0.085\textwidth}
      \resizebox{\textwidth}{!}{\includegraphics{images/cms.jpg}}\vspace{1em}
    \end{minipage}
    \begin{minipage}[c]{0.80\textwidth}
      \begin{center}
        \color{FireBrick}
        \textbf{\huge Background estimation in the search for \\[0.5em] $Z' \rightarrow ZH \rightarrow llbb$ at $\sqrt{s}$ = 13 TeV with the CMS detector} \\[1.8em]
        \noindent
        \color{Black}
        \textsc{\Large \underline{Yee-Shian Henry Tong}, Shin-Shan Eiko Yu, Raman Khurana} \\[0.8em]
        \noindent
            {\large %
              Department of Physics, National Central University, TaoYuan, Taiwan
            }
      \end{center}
    \end{minipage}
    \begin{minipage}[r]{0.105\textwidth}
      \resizebox{\textwidth}{!}{\includegraphics{images/ncu.jpg}}
    \end{minipage}
    
  \end{titlebox}
  
  %%%%%%%%%%%%%%%%%%%%%%%%%%%%%%%%%%%%%%%%%%%%%%%%%%%%%%%%%%%%%%%%%%%%%%%%%%%%%%%%
  %                             End of Header                                    %
  %%%%%%%%%%%%%%%%%%%%%%%%%%%%%%%%%%%%%%%%%%%%%%%%%%%%%%%%%%%%%%%%%%%%%%%%%%%%%%%%

  \vspace{-0.4em}

  \begin{multicols}{1}
    
    \begin{textbox0}
      \section*{\color{FireBrick} Abstract}
      
      We present the study of standard model (SM) background with the observed data for $Z' \rightarrow ZH \rightarrow llbb$ at $\sqrt{s}$ = 13 TeV with luminosity = 2.08 $fb^{-1}$. We define the signal region, and side-band region as a pure background control region by using $H (\rightarrow b\bar{b})$ jet mass distribution. In order to estimate the final background, we consider the $m_{ZH}$ Monte-Carlo (MC) mass spectrum in the signal and side-band region. A ratio $\alpha (m_{ZH})$ of the two allows a prediction of the mass spectrum in the signal region starting from the measured distribution in the side-band. Under the assumption that this extrapolation from the side-band to the signal region works in the same way both for data and MC, we can estimate the final background distribution by multiplying the $m_{ZH}$ mass spectrum observed in the side-band by this $\alpha$ ratio.

    \end{textbox0}

  \end{multicols}
  
  \vspace{-1.25em}

  \begin{multicols}{3}
    
    \begin{textbox}
      \section*{\color{FireBrick} Introduction}

      The aim of this analysis is to compare the predicted SM background with the observed data. We develop a data driven strategy for background estimation, to minimize the dependence on the MC. The signal region is defined in the range [105,135] GeV of the Higgs jet mass, in order to contain as much signal as possible. The side-band region is defined in the range [40,65] GeV and [145,$\infty$) GeV of the Higgs jet mass, as a pure background control region.
        \vspace{0.85em}
        \begin{center}
          \includegraphics[width=0.665\textwidth]{images/fey.jpg}
          \captionof{figure}{\small{\color{DarkRed} The Feynman diagram of the physics process of $Z' \rightarrow ZH \rightarrow llbb$.}}
          \label{fig:feyman}
        \end{center}
    \end{textbox}

    \begin{textbox}
      \section*{\color{FireBrick} Pre-selections}

      The following criteria are applied to select Z candidates and Higgs candidates.

      \begin{center}
        \begin{footnotesize}
          \begin{tabular}[t]{ | p{6cm} | p{14.5cm} | }
            \hline
            \bf Objects
            & \bf Requirements                         \\
            \hline
            \multirow{8}{*}{Muons}
            & Pass tracker muon ID                     \\
            & One of both pass high $p_T$ muon ID      \\
            & Isolation $<$ 0.2                        \\
            & $p_T$ $>$ 20 GeV                         \\
            & $|\eta|$ $<$ 2.4                         \\
            & At least one muon $p_T$ $>$ 50 GeV       \\
            & At least one muon $|\eta|$ $<$ 2.1       \\
            & Two muons have opposite charge           \\
            \hline
            \multirow{2}{*}{Z bosons}
            & 70GeV $<$ $M_{ll}$ $<$ 110 GeV           \\
            & $p_{Tll}$ $>$ 200 GeV                    \\
            \hline
            \multirow{3}{*}{AK8 jets}
            & $|\eta|$ $<$ 2.4                          \\
            & $p_T$ $>$ 200 GeV                         \\
            & $\Delta R$ between jet and lepton $>$ 0.8 \\
            \hline
            $Z'$
            & $M_{ZH}$ $>$ 500 GeV                      \\
            \hline
          \end{tabular}
          \captionof{table}{\small{\color{DarkRed} Summary of pre-selections in the analysis.}}
          \label{tab:select}
        \end{footnotesize}
      \end{center}
    \end{textbox}

    \begin{textbox}
      \section*{\color{FireBrick} Fit bias test}
      
      The fit bias test using toy MC shows that, due to the smaller number of events in data, the fitter result has a -2\% bias on the number of background events in the signal region.

      \begin{center}
        \includegraphics[page=5,width=0.949\textwidth]{images/aRanoorg.pdf}
        \captionof{figure}{\small{\color{DarkRed} The fit bias study in MC, with the Gaussian function fitting (blue).}}
        \label{fig:bias}
      \end{center}
      
    \end{textbox}
    
  \end{multicols}

  \vspace{-1.25em}

  \begin{multicols}{2}
    
    \begin{textbox2}
      \section*{\color{FireBrick} The Method of $\alpha$ ratio}

      The $\alpha$ ratio is a ratio of number of events of signal region and side band in $M_{ZH}$ distribution, as in Eq. (\ref{eq:alpha}). The distributions are fitted using Eq. (\ref{eq:zhfunc}). The alpha ratio as a function of $M_{ZH}$ is formed by dividing the two fit curves. Finally, we can obtain the data-drive background estimation using the $\alpha$ ratio, as in Eq. (\ref{eq:nbkg}).

      \begin{equation} \label{eq:alpha}
        \alpha(M_{ZH}) = \frac{N^{MC,bkg}_{signal}(M_{ZH})}{N^{MC,bkg}_{side}(M_{ZH})}
      \end{equation}
      \begin{equation} \label{eq:zhfunc}
        f(x) = p_0\exp(p_1x+\frac{p_2}{x})
      \end{equation}
      \begin{equation} \label{eq:nbkg}
        N_{bkg}(M_{ZH}) = N^{data}_{side}(M_{ZH})\times \alpha(M_{ZH})
      \end{equation}

      \begin{center}
        \begin{tabular}{ll}
          \includegraphics[page=4,width=0.47\textwidth]{images/alphaRwtData.pdf} &
          \includegraphics[page=5,width=0.47\textwidth]{images/alphaRwtData.pdf} \\
        \end{tabular}
        \captionof{figure}{\small{\color{DarkRed} The $M_{ZH}$ distribution in signal region (left) and side band (right) of MC.}}
        \label{fig:zpmass}
      \end{center}

      \begin{center}
        \begin{tabular}{ll}
          \includegraphics[page=6,width=0.47\textwidth]{images/alphaRwtData.pdf} &
          \includegraphics[page=7,width=0.47\textwidth]{images/alphaRwtData.pdf} \\
        \end{tabular}
        \captionof{figure}{\small{\color{DarkRed} The $\alpha$ ratio of MC as a function of $M_{ZH}$ (left) and the predicted background as a function of $M_{ZH}$ in signal region of data (right).}}
        \label{fig:predbkg}
      \end{center}
      
    \end{textbox2}

    \begin{textbox2}
      \section*{\color{FireBrick} Normalization}
      
      In order to use data itself to constrain the normalization, we use Eq. (\ref{eq:fitfunc}) to fit the jet mass without signal region, and calculate the area under the fit curve in signal region, as the normalization factor of estimated background.
      \vspace{1em}
      \begin{equation} \label{eq:fitfunc}
        f_{ErfExp}(x) = p_0\exp(p_1x)\frac{1+Erf(\frac{x-p_2}{p_3})}{2}
      \end{equation}
      
      \begin{center}
        \includegraphics[page=2,width=0.62\textwidth]{images/alphaRwtData.pdf}
        \captionof{figure}{\small{\color{DarkRed} The side band of $H (\rightarrow b\bar{b})$ jet mass distribution (black) and its fitting curve (blue).}}
        \label{fig:prmass}
      \end{center}
    \end{textbox2}
    
    \vspace{0.4em}

    \begin{textbox2}
      \section*{\color{FireBrick} Conclusions}

      \begin{item0}
      \item The fit curve for $M_{ZH}$ distributions is inconsistent at high $M_{ZH}$ mass region.
      \item The $\alpha$ ratio curve is inconsistent at high $M_{ZH}$ mass region.
      \item There is a -2\% fit bias which will be corrected when fitting data.
      \end{item0}

    \end{textbox2}

    \vspace{0.4em}
    
    \begin{textbox2}
      \section*{\color{FireBrick} Reference}
               [1] Andrea Mauri,\emph{Search for new exotic resonances in semileptonic ZH final state at CMS}, University of Pisa, 3 February 2014.
    \end{textbox2}

  \end{multicols}
  
\end{center}
\end{document}
