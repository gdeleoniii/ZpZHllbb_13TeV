\documentclass{beamer}
\mode<presentation>{
  \usetheme{Madrid}
  \usecolortheme{crane}
  %\setbeamertemplate{footline} % To remove the footer line in all slides uncomment this line
  %\setbeamertemplate{footline}[page number] % To replace the footer line in all slides with a simple slide count uncomment this line
  \setbeamertemplate{navigation symbols}{} % To remove the navigation symbols from the bottom of all slides uncomment this line
}
\setbeamercolor{itemize item}{fg=red}
\usepackage{listings}
\usepackage{caption}
\usepackage{setspace}
\usepackage{ragged2e} % For justifying text
\usepackage{multicol}
\usepackage{multirow}
\usepackage{graphicx} % Allows including images
\usepackage{booktabs} % Allows the use of \toprule, \midrule and \bottomrule in tables
\usepackage{enumerate}
\usepackage{amsmath}
\usepackage[final]{pdfpages}
\usepackage{hyperref}
\usepackage{pgffor}
\usepackage[makeroom]{cancel}
\usepackage{xcolor,colortbl}
\usefonttheme[onlymath]{serif}
\definecolor{RoyalYellow}{RGB}{250,218,94}
\graphicspath{{/afs/cern.ch/user/h/htong/www/}} % Specifies the directory where pictures are stored
\newcommand{\tabincell}[2]{\begin{tabular}{@{}#1@{}}#2\end{tabular}} 
\hypersetup{colorlinks=true}
\setbeamertemplate{title page}{
  \vbox{}
  \begingroup
  \centering{
    \usebeamercolor[fg]{titlegraphic}\inserttitlegraphic\par}\vskip0.5em
  \begin{beamercolorbox}[sep=8pt,center]{title}
    \usebeamerfont{title}\inserttitle\par%
    \ifx\insertsubtitle\@empty%
    \else%
    \vskip0.25em{%
      \usebeamerfont{subtitle}\usebeamercolor[fg]{subtitle}\insertsubtitle\par}%
    \fi%
  \end{beamercolorbox}%
  \vskip1em\par
  \begin{beamercolorbox}[sep=8pt,center]{author}
    \usebeamerfont{author}\insertauthor
  \end{beamercolorbox}
  \begin{beamercolorbox}[sep=8pt,center]{institute}
    \usebeamerfont{institute}\insertinstitute
  \end{beamercolorbox}
  \begin{beamercolorbox}[sep=8pt,center]{date}
    \usebeamerfont{date}\insertdate
  \end{beamercolorbox}
  \endgroup
  \vfill
}

% TITLE PAGE

% logo of my university
\titlegraphic{\includegraphics[width=2cm]{cms.jpg}\hspace*{8cm}~%
  \includegraphics[width=2cm]{ncu.jpg}
}

\title[]{Weekly Update} % The short title appears at the bottom of every slide, the full title is only on the title page
\author[Henry Tong]{Yee-Shian Henry Tong} % Your name
\institute[NCU]{ % Your institution as it will appear on the bottom of every slide, may be shorthand to save space
  National Central University \\ % Your institution for the title page
  \medskip
  \textit{NCU HEP Group Meeting} % Your email address
}
\date{\today} % Date, can be changed to a custom date

\begin{document}

\begin{frame}
  \vspace*{-1cm}
  \enlargethispage{1cm}
  \titlepage % Print the title page as the first slide
\end{frame}

% PRESENTATION SLIDES

\begin{frame}
  \frametitle{Introduction}
  \justifying
  \begin{itemize}
  \item I finish whole Chp 5 of my thesis(6 pages, all texts), which is talking about systematic uncertainties.
  \item I already have all skim samples with correct pu weight applied. I reproduce all the data/MC plots, there are no problem, even have a slightly slightly improve.
  \item I already finish all the systematic uncertainties study which relate to background shape and signal. (JES, QCD, PDF, bTag, PU)
  \end{itemize}
\end{frame}

\begin{frame}
  \frametitle{To do list}
  \justifying
  \begin{itemize}
  \item Get numbers of Syst. Unc. for bkg. norm. (change prmass model, fit deviation, fit goodness) (I of course can get them easily)
  \item draw bias distribution for the case of many events (put in my thesis).
  \item get uncertianties from the effect of substract minor bkgs.
  \item for JES, need to study the effect on norm. ? (maybe need to remind me do I miss something)
  \item actually need to take a look again the btagging unc (about the method of applying weight).
  \item also the other study which relate to btag should apply central value of btag scale factor.(but the effect seems very small)
  \end{itemize}
\end{frame}

\begin{frame}
  \frametitle{QCD scale Uncertainty}
  \justifying
  \begin{figure}[t]
    \centering
    \begin{tabular}{ll}
      \includegraphics[width=0.3\textwidth]{alphaPdfScaleResults/alpha_mur1Scale_ele_cat1.pdf} &
      \includegraphics[width=0.3\textwidth]{alphaPdfScaleResults/alpha_mur1Scale_ele_cat2.pdf} \\
      \includegraphics[width=0.3\textwidth]{alphaPdfScaleResults/alpha_mur1Scale_mu_cat1.pdf} &
      \includegraphics[width=0.3\textwidth]{alphaPdfScaleResults/alpha_mur1Scale_mu_cat2.pdf} \\
    \end{tabular}
    \caption{The QCD weights are stored in pdfscaleSysWeights[0-8], but here, the only used are [0-3], where is renormalization = 1 = const., and scale the factorization to 0.5,2. Multiply each weight to the event weight, so the the reconstructed mzh are weighted and are used to create alpha ratio.}
  \end{figure}
\end{frame}

\begin{frame}
  \frametitle{PDF scale Uncertainty}
  \justifying
  \begin{figure}[t]
    \centering
    \begin{tabular}{ll}
      \includegraphics[width=0.3\textwidth]{alphaPdfScaleResults/alpha_pdfScale_ele_cat1.pdf} &
      \includegraphics[width=0.3\textwidth]{alphaPdfScaleResults/alpha_pdfScale_ele_cat2.pdf} \\
      \includegraphics[width=0.3\textwidth]{alphaPdfScaleResults/alpha_pdfScale_mu_cat1.pdf} &
      \includegraphics[width=0.3\textwidth]{alphaPdfScaleResults/alpha_pdfScale_mu_cat2.pdf} \\
    \end{tabular}
    \caption{The 101 pdf weights are stored in pdfscaleSysWeights[9-109], where [9] is the central values. Multiply each weight to the event weight, so the the reconstructed mzh are weighted and are used to create alpha ratio.}
  \end{figure}
\end{frame}

\begin{frame}
  \frametitle{JES Uncertainty}
  \justifying
  \begin{figure}[t]
    \centering
    \begin{tabular}{ll}
      \includegraphics[width=0.3\textwidth]{alphaJetEnScaleResults/alpha_jetEnScale_ele_cat1.pdf} &
      \includegraphics[width=0.3\textwidth]{alphaJetEnScaleResults/alpha_jetEnScale_ele_cat2.pdf} \\
      \includegraphics[width=0.3\textwidth]{alphaJetEnScaleResults/alpha_jetEnScale_mu_cat1.pdf} &
      \includegraphics[width=0.3\textwidth]{alphaJetEnScaleResults/alpha_jetEnScale_mu_cat2.pdf} \\
    \end{tabular}
    \caption{\scriptsize The jet energy correction variables already saved in ntuples, call FATjetCorrUncUp/Down. In jet loop, eacj jet has its 4-momentum, multiply this to the 4-jet ( jet *= 1+FATjetCorrUncUp or jet *= 1-FATjetCorrUncDown) to get new 4-jet. Use this new 4-jet to reconstruct the mZH, and study the aplha ratio. Note that the signal and side band region still defined by using FATjetPRmassCorr[goodFATJetID], NOT newJet.M()}
  \end{figure}
\end{frame}

\begin{frame}
  \frametitle{b Tagging Uncertainty}
  \justifying
  \begin{figure}[t]
    \centering
    \begin{tabular}{ll}
      \includegraphics[width=0.3\textwidth]{alphabTagScaleResults/alpha_bTagScale_ele_cat1.pdf} &
      \includegraphics[width=0.3\textwidth]{alphabTagScaleResults/alpha_bTagScale_ele_cat2.pdf} \\
      \includegraphics[width=0.3\textwidth]{alphabTagScaleResults/alpha_bTagScale_mu_cat1.pdf} &
      \includegraphics[width=0.3\textwidth]{alphabTagScaleResults/alpha_bTagScale_mu_cat2.pdf} \\
    \end{tabular}
    \caption{\scriptsize After selecting the good jet (no matter it contains 1 or 2 bjets), I multiply the scale of this jet( according to pt/eta of jet, this is done by using the class provided by bTag POG. One thing is I m not sure the "flavour" is already be considered or not(I think yes: BTagEntry::FLAV\_B)) to event weight just like the PDF scale unc. Of course this is not the recommended method. Then, study alpha ratio.}
  \end{figure}
\end{frame}

\begin{frame}
  \frametitle{Signals}
  \justifying
  \begin{itemize}
  \item I wont show you the results of uncertainties mention above for signals. Because there are a lot of numbers, it is very trouble to show them here and I dont have that America timeto do that. But I will describe the way of how to do here.
  \item For QCD and PDF scale: results here http://htong.web.cern.ch/htong/pdfScaleResults/ (col 1: mZH, 2: QCD, 3: PDF). The numbers are uncertainties. when calculating signal efficiency, each pass event are weighted by one of those weights. Finally, I get 100 signal efficiencies (say, in case of PDF). The uncertainty value is get by (RMS of those 100 eff)/(central value of eff). Note that I didnt weight the total event here.
  \end{itemize}
\end{frame}

\begin{frame}
  \frametitle{Signals}
  \justifying
  \begin{itemize}
  \item For JES: results here http://htong.web.cern.ch/htong/jetEnScaleResults/ (col 1: mZH, 2:central, 3: Up, 4: Down). The numbers ARE signal efficiencies. So, change the 4-jet may change the number of pass event. But the effect is very very small (Ok actually there are no change, maybe I did it in wrong way)
  \item For bTagging: http://htong.web.cern.ch/htong/bTagResults/ (col 1: mZH, 2:central, 3: Up, 4: Down). The numbers ARE signal efficiencies. This is quite same as QCD/PDF scales. The pass event are weighted by the scale factor of the survive jet (from the class I mention before).
  \end{itemize}
\end{frame}

\begin{frame}
  \frametitle{Pile Up uncertainty}
  \justifying
  \begin{itemize}
  \item Results here http://htong.web.cern.ch/htong/puUncResults/ (col 1: mZH, 2:central, 3: Up, 4: Down). OK this is using three sets of MC which save different sets of pile up weight (hide in event weight). The 3 sets are come from varying the mean bias ccross section abt 5\% and generate by CMSSW. The main point is, these numbers are efficiencies. Each pass event are weighted by the event weight. This word "event weight" is the same teminology I use in previous slides, which includes (mcweight*pileup*kfactor*ewk). In previous uncertainty study, I didnt multiply this to my pass event.
  \end{itemize} 
\end{frame}

\begin{frame}
  \frametitle{Summary}
  \justifying
  Except btag (as its needs), no other studies above apply btag scale factor.
\end{frame}

\end{document}
