\documentclass{beamer}
\mode<presentation>{
  \usetheme{Madrid}
  \usecolortheme{crane}
  %\setbeamertemplate{footline} % To remove the footer line in all slides uncomment this line
  %\setbeamertemplate{footline}[page number] % To replace the footer line in all slides with a simple slide count uncomment this line
  \setbeamertemplate{navigation symbols}{} % To remove the navigation symbols from the bottom of all slides uncomment this line
}
\setbeamercolor{itemize item}{fg=red}
\usepackage{listings}
\usepackage{caption}
\usepackage{setspace}
\usepackage{ragged2e} % For justifying text
\usepackage{multicol}
\usepackage{multirow}
\usepackage{graphicx} % Allows including images
\usepackage{booktabs} % Allows the use of \toprule, \midrule and \bottomrule in tables
\usepackage{enumerate}
\usepackage{amsmath}
\usepackage[final]{pdfpages}
\usepackage{hyperref}
\usepackage{pgffor}
\usepackage[makeroom]{cancel}
\usepackage{xcolor,colortbl}
\usefonttheme[onlymath]{serif}
\definecolor{RoyalYellow}{RGB}{250,218,94}
\graphicspath{{/afs/cern.ch/user/h/htong/www/}} % Specifies the directory where pictures are stored
\newcommand{\tabincell}[2]{\begin{tabular}{@{}#1@{}}#2\end{tabular}} 
\hypersetup{colorlinks=true}
\setbeamertemplate{title page}{
  \vbox{}
  \begingroup
  \centering{
    \usebeamercolor[fg]{titlegraphic}\inserttitlegraphic\par}\vskip0.5em
  \begin{beamercolorbox}[sep=8pt,center]{title}
    \usebeamerfont{title}\inserttitle\par%
    \ifx\insertsubtitle\@empty%
    \else%
    \vskip0.25em{%
      \usebeamerfont{subtitle}\usebeamercolor[fg]{subtitle}\insertsubtitle\par}%
    \fi%
  \end{beamercolorbox}%
  \vskip1em\par
  \begin{beamercolorbox}[sep=8pt,center]{author}
    \usebeamerfont{author}\insertauthor
  \end{beamercolorbox}
  \begin{beamercolorbox}[sep=8pt,center]{institute}
    \usebeamerfont{institute}\insertinstitute
  \end{beamercolorbox}
  \begin{beamercolorbox}[sep=8pt,center]{date}
    \usebeamerfont{date}\insertdate
  \end{beamercolorbox}
  \endgroup
  \vfill
}

% TITLE PAGE

% logo of my university
\titlegraphic{\includegraphics[width=2cm]{cms.jpg}\hspace*{8cm}~%
  \includegraphics[width=2cm]{ncu.jpg}
}

\title[]{Weekly Update} % The short title appears at the bottom of every slide, the full title is only on the title page
\author[Henry Tong]{Yee-Shian Henry Tong} % Your name
\institute[NCU]{ % Your institution as it will appear on the bottom of every slide, may be shorthand to save space
  National Central University \\ % Your institution for the title page
  \medskip
  \textit{NCU HEP Group Meeting} % Your email address
}
\date{\today} % Date, can be changed to a custom date

\begin{document}

\begin{frame}
  \vspace*{-1cm}
  \enlargethispage{1cm}
  \titlepage % Print the title page as the first slide
\end{frame}

% PRESENTATION SLIDES

\begin{frame}
  \frametitle{Data and alpha ratio}
  \justifying
  \begin{figure}[t]
    \centering
    \begin{tabular}{ll}
      \includegraphics[page=1,width=0.3\textwidth]{rooFitDataResults/rooFit_forData_ele_cat1.pdf} &
      \includegraphics[page=1,width=0.3\textwidth]{rooFitDataResults/rooFit_forData_ele_cat2.pdf} \\
      \includegraphics[page=1,width=0.3\textwidth]{rooFitDataResults/rooFit_forData_mu_cat1.pdf} &
      \includegraphics[page=1,width=0.3\textwidth]{rooFitDataResults/rooFit_forData_mu_cat2.pdf} \\
    \end{tabular}
    \caption{ZH mass in MC in diffenent b-tag categories in electron channel and in muon channel.}
    \label{fig:mzhmc}
  \end{figure}
\end{frame}

\begin{frame}
  \frametitle{Data and alpha ratio}
  \justifying
  \begin{figure}[t]
    \centering
    \begin{tabular}{ll}
      \includegraphics[page=2,width=0.3\textwidth]{rooFitDataResults/rooFit_forData_ele_cat1.pdf} &
      \includegraphics[page=2,width=0.3\textwidth]{rooFitDataResults/rooFit_forData_ele_cat2.pdf} \\
      \includegraphics[page=2,width=0.3\textwidth]{rooFitDataResults/rooFit_forData_mu_cat1.pdf} &
      \includegraphics[page=2,width=0.3\textwidth]{rooFitDataResults/rooFit_forData_mu_cat2.pdf} \\
    \end{tabular}
    \caption{ZH mass of side band and its fit in data in diffenent b-tag categories in electron channel and in muon channel.}
    \label{fig:mzhdata}
  \end{figure}
\end{frame}

\begin{frame}
  \frametitle{Data and alpha ratio}
  \justifying
  \begin{figure}[t]
    \centering
    \begin{tabular}{ll}
      \includegraphics[page=3,width=0.3\textwidth]{rooFitDataResults/rooFit_forData_ele_cat1.pdf} &
      \includegraphics[page=3,width=0.3\textwidth]{rooFitDataResults/rooFit_forData_ele_cat2.pdf} \\
      \includegraphics[page=3,width=0.3\textwidth]{rooFitDataResults/rooFit_forData_mu_cat1.pdf} &
      \includegraphics[page=3,width=0.3\textwidth]{rooFitDataResults/rooFit_forData_mu_cat2.pdf} \\
    \end{tabular}
    \caption{ZH mass of signal region in data and the predicted background in diffenent b-tag categories in electron channel and in muon channel.}
    \label{fig:mzhdata}
  \end{figure}
\end{frame}

\begin{frame}
  \frametitle{Data and alpha ratio}
  \justifying
  \begin{figure}[t]
    \centering
    \begin{tabular}{ll}
      \includegraphics[page=4,width=0.3\textwidth]{rooFitDataResults/rooFit_forData_ele_cat1.pdf} &
      \includegraphics[page=4,width=0.3\textwidth]{rooFitDataResults/rooFit_forData_ele_cat2.pdf} \\
      \includegraphics[page=4,width=0.3\textwidth]{rooFitDataResults/rooFit_forData_mu_cat1.pdf} &
      \includegraphics[page=4,width=0.3\textwidth]{rooFitDataResults/rooFit_forData_mu_cat2.pdf} \\
    \end{tabular}
    \caption{Jet mass and its fit, using exponential function, in data in diffenent b-tag categories in electron channel and in muon channel. Compare to this http://htong.web.cern.ch/htong/main.pdf in page 59, which the distributions are fit by error exp function, there is no so much difference.}
    \label{fig:mjetdata}
  \end{figure}
\end{frame}

\end{document}
